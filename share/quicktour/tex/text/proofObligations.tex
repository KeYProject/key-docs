\section{Provable properties}
\label{sec:provableProp}
% evtl. Benajmins Diss zitieren
In the following, the ideas behind the various options for verification
are described informally. A formal description of the generated proof
obligations is contained in~\cite{KeYBook2016}. For
further details on the mapping between JML specifications and the
formulas of the JavaDL logic used in \KeY, please
consult~\cite{Engel05}. 

Examples of usage within the context of the case study in this
tutorial are described in Sect.~\ref{sec:application}.

\subsection{Informal Description of Proof Obligations}
The current implementation of the \kp{} generates two kinds of proof 
obligations: functional contracts and dependency 
contracts\footnote{For a detailed description of both kinds of contracts, see~\cite{Weiss2011}.}.
Method contracts describe the behavior of a method.
Properties covered by method contracts include:
\begin{itemize}
 \item \emph{properties for method specifications:} we show that a method
  \emph{fulfills} its method contract,
\item \emph{properties for class/object specifications:} we show that a method
  \emph{preserves} invariants of the object on which the call is invoked.

\end{itemize}

For the verification of programs, \emph{dependency contracts} are used to be able to rely on the 
properties of, e.g. the object invariant, in parts of the proof where 
method calls are invoked or other instructions are performed which change the memory locations on the heap. 
If these method calls only change memory locations on the heap which 
do not affect those location on which the invariant at most depends on, 
it is still possible to use the stated properties of the invariant after such a method call.
However, if the method calls also change location on the heap on which the invariant depends on, 
it is not possible to rely on those properties anymore and it has to be proven that the invariant 
still holds. For a more detailed description, see~\cite{Weiss2011}.


\subsubsection{The Logic in Use}
\label{sec:logc}

In this section, we make a short excursion to the formalism underlying
the \kt. As we follow a deduction-based approach towards software
verification, logics are the basic formalism used. More precisely,
we use a
typed first-order dynamic logic called JavaCardDL. 

We do not intend here to give a formal introduction into the
logic used, but we explain the intended meaning of the formulas. Furthermore, we
assume that the reader has some basic knowledge of classical
first-order logic. 

In addition to classical first-order logic, dynamic logic has two
additional operators called modalities, namely the diamond
$\dldia{\cdot}{\cdot}$ and box $\dlbox{\cdot}{\cdot}$ modalities. Their
first argument is a piece of JavaCard code and the second argument
an arbitrary formula. Let $p$ be a program and $\phi$ an arbitrary
formula in JavaCardDL. Then 
\begin{itemize}
  \item $\dldia{p}{\phi}$ is a formula in JavaCardDL which holds iff
    program $p$ terminates \textbf{and} in its final state formula $\phi$
    holds.  
  \item $\dlbox{p}{\phi}$ is a formula in JavaCardDL which holds iff
    \textbf{if} program $p$ terminates \textbf{then} in its final
    state formula $\phi$ holds.
\end{itemize}

The notion of a \emph{state} is a central one. A state can be
seen as a snapshot of the memory when running a program. It
describes the values of each variable or field. A formula in
JavaCardDL is evaluated in such a state. 

Let $i,j$ denote program variables. Some formulas in JavaCardDL:

\begin{itemize}
\item The formula \[
  i \keyeq 0 \keyimplies \dldia{i = i + 1;}{i \keygt 0}
  \] is a formula in JavaCardDL. The formual states:

  \begin{quote}
    If the value of $i$ is $0$, then the program $i = i + 1;$ terminates
    \emph{and} in the final state (the state reached after executing the
    program), the program variable $i$ is greater than $0$.
  \end{quote}

  The diamond operator states implicitly that the program must
  terminate normally, i.e., no infinite loops and recursions and no uncaught
  exceptions occur.

  Replacing the diamond in the formula above by a box \[
     i \keyeq 0 \keyimplies \dlbox{i = i + 1;}{i \keygt 0}
  \]
  changes the termination aspect and does not require that the program
  terminates, i.e., this formula is already satisfied if in each state
  where the value of $i$ is $0$ and \emph{if} the program $i = i + 1;$
  terminates \emph{then} in its final state $i$ is greater than $0$.

\item A typical kind of formula you will encounter is one with an update
  in front like \[ 
      \applyUp{\parUp{\elUp{i}{a}}{\elUp{j}{b}}}{\dldia{tmp = i; i =
          j; j = tmp;}{i \keyeq b \keyand j \keyeq a}}
  \] Intuitively, an update can be seen as an assignment, the two
  vertical strokes indicate that the two assignments $a$ to $i$ and
  $b$ to $j$ are performed in parallel (simultaneously). The formula
  behind the update is then valid if in the state reached executing
  the two ``assignments'', the program terminates (diamond!) and
  in the final state the content of the variables $i$ and $j$ have
  been swapped.
\end{itemize}

\subsubsection{Sequents}
\label{sec:sequents}

Deduction with the \kp\ is based on a sequent calculus for a dynamic
logic for JavaCard (JavaDL)~\cite{KeYBook2007,Beckert01}.

A sequent
has the form $\phi_1, \ldots,\phi_m\;\vdash\;\psi_1,\ldots,\psi_n\;
(m,n \geq 0)$, where the $\phi_i$ and $\psi_j$ are JavaDL-formulas.
The formulas on the left-hand side of the sequent symbol $\;\vdash\;$
are called the {\it antecedent} and the formulas on the right-hand side
are called the {\it succedent}. The semantics of such a sequent is the same as
that of the formula $(\phi_1\land\ldots\land \phi_m) \to (\psi_1
\lor\ldots\lor \psi_n)$.

% 
\subsection{Proof-Obligations}

In general, a proof obligation is a formula that has to be proved
valid. When we refer to a proof obligation, we usually mean the
designated formula occurring in the root sequent of the proof.
%
A method contract for a method $m$ of a class $C$ consists in general
of a 
\begin{description}
  \item[precondition $pre$] describing the method-specific\footnote{Additional conditions stem from invariants.}
    conditions which a caller of the method has to fulfill before calling the
    method in order to be guaranteed that the
  \item[postcondition $post$] holds after executing the method and
    that the 
  \item[assignable/modifies clause $mod$] is respected. This means
    that at most the locations described by $mod$ are modified in the
    final state. In addition, we have a
  \item[termination marker] indicating if termination of the method is
    required. Termination required (total correctness) has the termination
    marker \jn{diamond}, i.e.\ the method must terminate when the
    called in a state where the precondition is fulfilled. The marker
    \jn{box} does not require termination (partial correctness), i.e.,
    the contract must only be fulfilled if the method terminates.
\end{description}

In addition, each object $O$ has a possibly empty set of invariants
$inv_{O}$ assigned to them.

For the general description, we refer to this general kind of
contract. Mapping of JML specifications to this general contract notion
is slightly indicated in Sect.~\ref{sec:application}. More details can
be found in~\cite{KeYBook2016}.

\subsection{Application to the Tutorial Example}
\label{sec:application}

Now we apply the described proof obligations to the tutorial
example. First  we demonstrate the generation of proof obligations,
then we show how these can be handled by the \kp. Please make sure
that the default settings of the \kp{} are selected (see Chapter~%
\ref{sec:prover}) and the maximum number of automatic rule applications is
5000. Be aware that the names of the proof rules and the structure of
the proof obligations may be subject to change in the future.

\subsubsection{Method Specifications}

\paragraph{Normal Behavior Specification Case.}

In the left part of the \pob, expand the directory \jn{paycard}.
Select the class \jn{PayCard} and then the method
\jn{isValid}. This method is specified by the JML
annotation
\begin{verbatim}
   public normal_behavior
     requires true;
     ensures result == (unsuccessfulOperations<=3); 
     assignable \nothing;
\end{verbatim}

This JML method specification treats the \texttt{normal\char`\_behavior}
case, i.e., a method satisfying the precondition (JML boolean
expression following the \texttt{re\-quires} keyword) \emph{must not}
terminate abruptly throwing an exception. Further each method
satisfying the precondition must
\begin{itemize}
\item terminate (missing diverges clause),
\item satisfy the postcondition -- the JML boolean expression after
  the \texttt{ensures} keyword, and
\item only change the locations expressed in the \texttt{assignable}
  clause; here: must not change any location. The assignable clause is
  actually redundant in this concrete example, as the method is
  already marked as \texttt{pure} which implies
  \verb+assignable \nothing+.
\end{itemize}

Within \KeY, you can now prove that the implementation satisfies the
different aspects of the specification, i.e., that if the precondition
is satisfied then the method actually terminates normally and
satisfies the postconditon and that the assignable clause is
respected. In addition it is also proven that the object invariants of 
the object on which the method call is invoked are preserved.

The contracts pane in the \prm{} window offers only one proof obligation:
\jn{JML normal\_behavior opertaion contract}. 
It summarizes all parts of the specification which will be considered in the proof.

The selected contract says that a call to this method, if the object 
invariant holds  (\mea{pre} \jn{self.\textless{}inv\textgreater}), always
terminates normally and that the \jn{return} value is true iff
the parameter \jn{un\-success\-ful\-Ope\-ra\-tions} is $\leq 3$, 
Additionally, the object invariant holds after the method call, 
therefore it is preserved.
The sequent displayed in the large prover window after loading the proof
obligation exactly reflects this property.
We confirm the selection by pressing the \tn{Start proof} button.
  
Start the
proof by pushing the \tn{Start}-button (the one with the green
``play'' symbol). The proof is closed automatically by the
strategies. It might be necessary that you have to push the button
more than once if there are more rule applications needed than you
have specified with the ``Max. Rule Applications'' slider.

\paragraph{Exceptional Behavior Specification Case.}
An example of an \texttt{exceptional\_behavior} specification case can be found
in the JML specification of \jn{PayCard::charge(int amount)}. The exceptional case reads
\begin{verbatim}
  public exceptional_behavior
      requires amount <= 0;
\end{verbatim}

This JML specification is for the exceptional case. In contrast to the
\texttt{normal\_behavior} case, the precondition here states under which
circumstances the method is expected to terminate abruptly by throwing
an exception. 

Use the \prm\ (either by clicking onto the button 
\includegraphics[height=2ex]{../figures/proofManagementButton} 
or \meb{File}{Proof Management}).
Continue as before, but this time, select the method
\jn{PayCard::charge(int amount)}. In contrast to
the previous example, the contracts pane offers you three contracts: two for the
normal behavior case and one for the exceptional case. As we want to
prove the contract for the exceptional case select the contract named:
\emph{JML exceptional\_behavior operation contract}. 

The \KeY\ proof obligation for this specification requires that if the
parameter \jn{amount} is negative or equal to $0$, then the method
throw a \jn{Illegal\-Argument\-Exception}.

Start the proof again by pushing the \tn{run
  strategy}-button. The proof is closed automatically by the
strategies.

\paragraph{Generic Behavior Specification Case.}

The method specification for method \jn{PayCard::createJuniorCard()} is:
\begin{verbatim}
   ensures \result.limit==100;
\end{verbatim}
This is a lightweight specification, for which \KeY\ provides a proof
obligation that requires the method to terminate (maybe abruptly) and
to ensure that, if it terminates normally, the \jn{limit} attribute of
the result equals $100$ in the post-state. 
By selecting the
method and choosing the \emph{JML operation
  contract} in the \ctCfg, an appropriate JavaDL
formula is loaded in the prover. The proof can be closed automatically
by the automatic strategy.

\subsubsection{Type Specifications}

The instance invariant of type \jn{PayCard} is
\begin{verbatim}
    public invariant log.\inv 
    && balance >= 0 && limit > 0 
    && unsuccessfulOperations >= 0 && log != null;
\end{verbatim}

The invariant specifies that the \jn{balance} of a paycard must be non-negative, that it must be possible to charge it with some money (\jn{limit > 0}) and that the number of \jn{unsuccessfulOperations} cannot be negative. Furthermore, the invariant states that the log file, which keeps track of the transactions, must always exist (\jn{log != null}) and that the instance referred to by \jn{log} has to satisfy the instance invariant of its own class type (\verb+log.\inv+).  

The method \jn{PayCard::charge()} must preserve this
invariant unless it does not terminate. 
This proof obligation is covered by proving that 
\verb+self.<inv>+ must 
be satisfied in the precondition and must hold in the postcondition.
This has to be covered by all specification cases and is included in 
the specification in the \ctCfg (\verb+self.<inv>+ 
in the pre- and in the postcondition).

There is one exception: if a method is annotated with the keyword \jn{helper},
the proof obligation will not include the invariants. An example for such a method is \jn{LogRecord::setRecord()}.
If a method is declared as helper method, the invariant of the object on which the method is called
does not have to hold for its pre- and postcondition. For more details, see~\cite{Weiss2011}.
The advantage is a simpler proof obligation; however, the disadvantage is that
in order to show that the method fulfills its contract, it is not possible
to rely on the invariants.

In addition, invariants can also be annotated with an \texttt{accessible} clause.
In this clause, all fields and locations on which the invariant depends have to be included.
For example, the invariant of class \jn{LogRecord} is 
\begin{verbatim}
 !empty ==> (balance >= 0 && transactionId >= 0);
\end{verbatim} and it is annotated with the accessible clause
\begin{verbatim}
 accessible \inv: this.empty, this.balance, 
 transactionCounter, this.transactionId;
 \end{verbatim}

This means the invariant of \jn{LogRecord} depends at most on the fields or 
locations given in the \jn{accessible} clause. If
one of these fields is changed by a method, it has to be proven that the invariant still holds.
The accessible clause can simplify proofs because in cases where the heap is changed 
but not the mentioned fields and the validity of the invariant is proven before, 
we know that the invariant still holds.


\subsubsection{Proof-Supporting JML Annotations}

In \KeY, JML annotations are not only used to generate proof obligations but
also support proof search. An example are loop invariants. In our scenario,
there is a class \jn{LogFile}, which keeps track of a number of recent
transactions by storing the balances at the end of the transactions. Consider
the constructor \jn{LogFile()} of that class.
To prove the \jn{normal\_behavior} specification proof obligation of the method, one needs to
reason about the incorporated \jn{while} loop. Basically, there are two
ways do this in \KeY: use induction or use loop invariants. In
general, both methods require interaction with the user during proof
construction. For loop invariants, however, \emph{no interaction} is needed if
the JML \jn{loop\_invariant} annotation is used.
In the example, the JML loop invariant indicates that the 
array \jn{logArray} contains newly created entries (\jn{fresh(logArray[x])}) and that 
these entries are not \jn{null} in the array from the beginning up
to position \jn{i}:
\begin{lstlisting}
 	/*@ loop_invariant
	  @ 0 <= i && i <= logArray.length &&
	  @ (\forall int x; 0 <= x && x < i; logArray[x] != null && \fresh(logArray[x]))
	  @ && LogRecord.transactionCounter >= 0;
	  @ assignable logArray[*], LogRecord.transactionCounter;
	  @ decreases logArray.length - i; 
	  @*/	   
	while(i<logArray.length){ 
	    logArray[i++] = new LogRecord();
	}
\end{lstlisting}

If the annotation had been skipped, we would have been asked during the proof
to enter an invariant or an induction hypothesis. With the annotation no
further interaction is required to resolve the loop.


Select the contract \emph{JML normal
  behavior operation contract} \jn{LogFile}\hspace{0pt}\jn{::}\hspace{0pt}\jn{LogFile()}.

Choose \emph{Loop treatment} \emph{None} and start the prover.  
When no further rules can be applied automatically,
select the while loop including the leading updates, press the mouse
button, select the rule \emph{loopInvariant} and then \emph{Apply Rule}. This rule makes use
of the invariants and assignables specified in the source
code. 
Restart the strategies and run them until all goals are closed.

As can be seen, \KeY\ makes use of an extension to JML, which is that
\emph{assignable} clauses can be attached to loop bodies, in order to
indicate the locations that can at most be changed by the body. Doing
this makes formalizing the loop invariant considerably simpler as the
specifier needs not to add information to the invariant specifying all
those program variables and fields that are not changed by the loop.
Of course one has to check that the given assignable clause is
correct, this is done by the invariant rule. We refer
to~\cite{KeYBook2016} for further discussion and pointers on this
topic.

% \subsubsection{Using the \KeY-plugin for Eclipse}
% 
% \emph{\Large This section is currently out-of-date as the eclipse
%   plug-ins are undergoing restructuring. The principal approach is
%   still valid.}
% 
% This section will give a quick overview on the visualization
% features added by the \KeY-plugin for Eclipse. We will
% assume that the plugin has already been installed as described
% above. Start Eclipse and open the \emph{PayCard} project using
% the \emph{Import} dialog from the \emph{File} menu. The 
% \emph{paycard} directory should appear on the right hand side.
% Now open the proof visualization by selecting \emph{Other} in the
% \emph{Show View} submenu inside the \emph{View} menu. Once there
% select \emph{Proof Visualization} in the \emph{KeY} branch 
% and click \emph{OK}. Now it is time to actually open
% one of the classes, in this example we will use LogFile. 
% 
% Open the \kt\ by clicking on the \KeY-logo in the toolbar.
% As before select the \emph{PayCard} project and mark the 
% \emph{normal\_behavior speccase} for the method 
% \emph{getMaximumRecord} in the \emph{LogFile} class. Now start
% the proof. A number of open goals will remain but this time
% we won't deal with them, our focus is on the Eclipse plugin. 
% 
% It is time to take a closer look at the visualization
% options in Eclipse. Return to it and press the \emph{Show
% Execution Traces} button from the \emph{Proof Visualization}
% view. A new window should pop up with a number of execution
% traces available. Checking the \emph{Filter uninteresting traces}
% option hides those traces that appear to be irrelevant to the
% understanding of the current proof. In this case it should leave
% you with a single trace. Mark it and click \emph{OK}. Now you will
% actually see the execution trace of the selected node in the
% \emph{Proof Visualization} view. Additionally all executed statements,
% except for the last one, are highlighted in yellow inside the Java
% editor. In case of an exception the last statement is highlighted in
% dark red, otherwise in dark yellow. You can navigate through the trace
% using the buttons \emph{Step Into} and \emph{Step over}. The first one
% allows you to mark the next executed statement while the last one jumps
% over substatements and method calls. By right-clicking on a branch you
% can also choose to go into it. To return to the main execution trace
% press the \emph{Home} button. Pushing the \emph{Mark All Statements}
% button remarks all statements of the trace in the Java editor. If you
% want to clear all markers you can press the red cross. It is possible
% to receive more information on single statement, like the node at which
% the statement was executed, by moving the mouse over the marker bar
% left of the Java editor.
% 
% In case Eclipse is not available you can use a more rudimentary visualization
% built directly in the \kt. You can access it by right-clicking on a node
% and selecting \emph{Visualize}. This opens a new window with a list of traces.
% Again you have the chance to \emph{Filter uninteresting traces} and you get
% to see the trace in a tree-like structure. Statements that produced exceptions
% are highlighted in red.
% 
% The visualization options presented above concentrate on the symbolic
% execution. They allow an intuitive way for analyzing the current proof
% branch in a way that is similar to classic debuggers.


%%% Local Variables: 
%%% mode: latex
%%% TeX-master: "quicktour"
%%% End: 

% LocalWords:  postcondition
