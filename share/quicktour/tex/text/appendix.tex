\appendix

\section{List of Menu Options}
\label{app:menuOptions}

In the following, we describe some menu items available in the main menu
of the \kp.

\begin{description}

\item[\mea{File}] File related actions
  \begin{description}
  \item[\meb{}{Load Example}:] Opens a file browser with included example files.
  \item[\meb{}{Load}:] Loads a problem or proof file; selecting a
    directory opens the proof management window with the generated
    proof obligation for the chosen specification language
  \item[\meb{}{Reload}:] Reloads last file loaded.
  \item[\meb{}{Edit last opened file}:] If a default external editor is set, 
  this action opens the last opened file in the default external editor.  

  \item[\meb{}{Save}:] Saves the current selected proof. Note, that if
    there are several proofs loaded (see the upper left pane) only the one
    currently worked on is saved. 
   \item[\meb{}{Save as Proof Bundle}:] Saves the current selected proof as a bundle containing not only the proof itself but all source file required to reload it.
   \item[\meb{}{Quicksave}:] Saves the current selected proof to a temporary file.
   \item[\meb{}{Quickload}:] Loads the last quicksaved proof.
    
  \item[\meb{}{Proof Management}:] Allows browsing through the
    available proof obligations.
  \item[\meb{}{Load User-defined Taclets}:] Allows to activate and deactivate
    theories given as taclet collection in a \texttt{.key} file.
  \item[\meb{}{Prove}:]
  \begin{description}
    \item[\meb{}{User-Defined Taclets}:] Loads user-defined taclets 
    and generates the corresponding proof obligation.
      
    \item[\meb{}{\KeY{}'s Taclets}:] Creates proof obligations for some selected taclets.

    \item[\meb{}{Taclets using the batch mode}:]
    \end{description}
    
%   
%   \item[\meb{}{Reload Last Problem}:] 
% 		Reloads the problem you are currently working on.

  \item[\meb{}{Recent Files}:] List the last few loaded files (if
    they are still present).
    
  \item[\meb{}{Exit}:] Quits the \kp\ (be warned: the current
    proof is lost!).
  \end{description}

\item[\mea{View}] Settings influencing the look of the user interface
  \begin{description}
  \item[\meb{}{Use Pretty Syntax}:] If set, infix notation for functions and predicates are used.
  \item[\meb{}{Use Unicode Symbols}:] If set, unicode symbols instead of ASCII symbols are used for logical symbols.
  \item[\meb{}{Use Syntax Highlighting}:] Use syntax highlighting in the sequent.
  \item[\meb{}{Term labels}:] Toggle the visibility of different kinds of term labels. As explained in Sec.~\ref{sec:prover}, these do not change the semantics of the sequent but
  carry additional information that helps guide the automatic proof search and/or user.
  
  \item[\meb{}{Hide Package Prefixes}:] Print unqualified class names in the sequent instead of qualified ones.
  
  \item[\meb{}{Show Tooltips in Sequent View}:] Show a tooltip with additional information about the currently selected term.
  \item[\meb{}{Show Tooltips in Source View}:] Show a tooltip with additional information about the currently selected term.

  \item[\meb{}{Font size}] Changes the font size of the sequent and source views.

  \item[\meb{}{ToolTip options}:] Configures the tooltip shown when
    hovering over a taclet in the list of applicable taclets.
  \item[\meb{}{Visual Node Diff}:] Opens a new window which allows 
  to view the  difference between two chosen proof nodes.
  
  \item[\meb{}{Select goal}:] Select another goal in the current proof.
  \item[\meb{}{Heatmap}:] Configure heatmaps. This feature---turned off by default---highlights the most recent changes in the sequent.
  \item[\meb{}{Layout}:] Save and load the current UI layout.
  \item[\meb{}{Exploration}:] Options pertaining to proof exploration.
  \begin{description}
  	\item[\meb{}{Exploration mode}:] Enables proof exploration. This allows you to rewrite, add, or remove formulas on the sequent.
  	\item[\meb{}{Hide justification}:] Editing the sequent creates a {\em justification branch}, in which you must show that your edit does not compromise the correctness of the proof. This option allows you to hide these justification branches.
  \end{description}
  

  \end{description}

\item[\mea{Proof}] Proof specific options
  \begin{description}
  \item[\meb{}{Start}:] Run the proof (semi-)automatically w.r.t.\ 
	to current strategy options.

  \item[\meb{}{Undo Last Rule Application}:] Undo one proof step.
  \item[\meb{}{Prune proof}:] Undo all proof steps below the currently selected node.

  \item[\meb{}{Abandon Proof}:] Quits the currently active
    proof. All other loaded problems will stay in the \kp.
  \item[\meb{}{Search in proof tree}:] Opens a textfield in the proof pane, which allows to 
  search for a string in the proof tree.
  \item[\meb{}{Search in sequent view}:] Opens a window with a textfield in the lower-left 
  corner of the current goal pane, which allows to search for a string in the sequent view.
  
  \item[\meb{}{Show used contracts}:]

  \item[\meb{}{Show Active Taclet Options}:] Shows the taclet options
    chosen for the current proof.

  \item[\meb{}{Show All Active Settings}:] Opens a window displaying all settings used in the current proof.

  \item[\meb{}{Show Proof Statistics}:] Shows some general statistics
    about the proof size and interactive steps.

  \item[\meb{}{Show Known Types}:] Lists all types present in the
    current proof environment. 
    
  \item[\meb{}{Search for Counterexample}:] Searches for a counterexample to the current proof. This requires an SMT solver to be installed.
  
  \item[\meb{}{Generate Testcases}:] Generate test cases for open goals in the proof. This requires an SMT solver to be installed.

  \end{description}
  
\item[\mea{Options}] General options


  \begin{description}
  	
  \item[\meb{}{Show Settings}:] Open a settings dialog to configure \KeY's appearance.
  \item[\meb{}{Show Taclet Options}:] In the following, each taclet
    option is described briefly. The respective default settings are
    given in parenthesis. The meaning of all settings is beyond the
    scope of this quicktour. Please use the default settings unless
    you know what you are doing.
    Note that this list is not complete.

    \begin{description}
 \item[JavaCard:] (\textsf{off}) There are two values for this option:
 \textsf{on} and \textsf{off}. Switches all
 taclets axiomatising JavaCard specific features like
 transaction on or off. 

    \item[assertions:] (\textsf{on}) There exists are
      different values for this option
      \begin{description}
        \item[\textsf{on}] evaluates assert statements and raises an
          \jn{As\-ser\-tion\-Exception} if the condition evaluates to
          false. This behaviour models the behaviour of the Java
          virtual machine with assertions enabled globally.
        \item[\textsf{off}] skips evaluation of assert statement. In
          particular, the arguments of the assert statements are not
          evaluated at all. This behaviour models the behaviour of the
          Java virtual machine with assertions disabled globally.
        \item[\textsf{safe}] using this option ensures that the shown property
          is valid no matter if assertions are globally enabled or
          disabled. Proofs with this option are typically harder. 
      \end{description}
      Please note: There is no support other than option \textsf{safe}
      for enabling or disabling assertions package or class wise.

     \item[initialisation:] (\textsf{disableStaticInitialisation})
    Specifies whether static initialization should be considered.

    \item[intRules:] 
      (\textsf{arithmeticSemanticIgnoringOF}) Here you can choose
      between different semantics for Java integer arithmetic (for
      details
      see~\cite{Schlager02,SchlagerPhD2007,KeYBook2007}). Three
      choices are offered:

      \begin{description}
      \item[\textsf{javaSemantics}] (Java semantics): Corresponds
        exactly to the semantics defined in the Java language
        specification. In particular this means, that arithmetical
        operations may cause over-/underflow. This setting provides
        correctness but allows over-/underflows causing unwanted
        side-effects.
	This corresponds to the \texttt{code\char`\_java\char`\_math}
	macro in JML.
      \item[\textsf{arithmeticSemanticsIgnoringOF}] (Arithmetic
        semantics ignoring overflow, default): Treats the primitive
        finite Java types as if they had the same semantics as
        mathematical integers with infinite range. Thus this setting
        does not fulfil the correctness criteria.
	This corresponds to the \texttt{code\char`\_bigint\char`\_math}
	macro in JML.
      \item[\textsf{arithmeticSemanticsCheckingOF}] (Arithmetic
        semantics prohibiting overflow): Same as above but the result
        of arithmetical operations is not allowed to exceed the range
        of the Java type as defined in the language
        specification. This setting not only enforces the java
        semantics but also ascertains that no overflow occur.
	This corresponds to the \texttt{code\char`\_safe\char`\_math}
	macro in JML.
      \end{description}

    \item[programRules:] (\textsf{Java}) Changes between different
      program languages\footnote{Ensure that \tn{Java} is selected.}.

     \item[model fields] The semantics of model fields is given by 
     the \texttt{represents} clause in the JML specification. 
     The setting of this option decides how the represents clauses are handeled.
     It has two possible values \textsf{treatAsAxiom} 
     and \textsf{showSatisfiability}:
     \begin{description}
      \item[treatAsAxiom:] Represents clauses are seen as axioms. 
      If this option is set the satisfiability of the represents 
      clauses is not shwon and therefore it may introduce inconsistent
      specifications, e.g., he following contradictory JML clause
      will not be rejected:\\
      \texttt{//@ represents modelField == modelField + 1;}
      \item[showSatisfiability:] For every expansion of the represents 
      clause, the satisfiability of the definition has to be shown.
      Cross-definition inconsistencies can still be
      formulated, however:\\
	\texttt{//@ represents modelField1 == modelField2;}\\
	 \texttt{//@ represents modelField2 == modelField1 + 1;}
     \end{description}


     
     \item[runtime exceptions] There are two possible ways for the \kp\ in
     handling runtime exceptions -- \textsf{ban} or \textsf{allow}.
     \begin{description}
      \item[ban:] If runtime exceptions are banned, \KeY\ treats the occurence of
      runtime exceptions as irrecoverable program failure. Setting this option
      results in smaller proofs and is complete for defensive programmed programs, i.e.
      programs which do not intentionally use corner cases. 
     
      \item[allow:] If runtime exceptions are allowed, \KeY\ treats 
      runtime exceptions as defined in the Java language specification, i.e.
      implicit runtime exceptions\footnote{With implicit exceptions, we mean 
      exceptions not explicitely programmed by the developer using 
      \texttt{throw new java.lang.exception}.}
      are raised and therefore such exceptions have to be considered in the proof.
      Setting this option results in larger proofs.
     \end{description}

    \end{description}

    The current setting of the taclet options can be viewed by choosing
    \meb{Proof}{Show Active Taclet Options}.
   \item[\meb{}{Minimize Interaction}] If this option is used and the automtaic strategy is used, 
   \KeY{} tries to minimize user interaction. That means that for example, 
   if the \kp{} is able to find instantiations by itself, the user is not asked to provide them.
   \item[\meb{}{Right Click for Macros}]
   \item[\meb{}{One Step Simplification}] In the \kp\ one step simplification 
   is a mechanism to automatically apply several simplifying and normalization rule applications to the sequent. 
   For the user these rule applications are aggregated 
   into one visible rule application \textsf{One Step Simplification} 
   in the proof tree.
   Setting this option often leads to simpler sequents and results in finding a proof faster,
   but the user lacks
   transparency of the proof, because the rule applications of the one step simplifier
   are not shown in very detail compared to all other rule applications in the \kp.
  
  \item[\meb{}{Show SMT Solver Options}:] This option allows you to choose
    one or more external decision procedures that can be invoked during
    proofs and to set options for each external solver seperately. 
    There is a native interface to 
    \textsf{Simplify}. A variety of other provers
    \textsf{CVC3},  \textsf{Yices}, and \textsf{Z3}
    are directly supported via SMTLIB~\cite{RanTin-SMTLIB}. 
    In addition, translations of taclets to the SMTLIB language
    can be written to a text file (\textsf{Taclet
      Translation}) to be loaded by any SMT prover. 
    There are further options on the set of taclets to translate.
  \end{description}
  
  \item[\mea{Interaction Logging}:] These options configure the interaction log in the left pane, which logs all user interactions on the loaded proof.
  
  \item[\mea{Origin Tracking}:] These options configure the origin tracking feature. By default, a JavaDL term's origin in the JML specification is tracked throughout the proof and highlighted in the source pane when mousing over the term.
  

\end{description}
\section{List of all Strategy Tab Settings}
\label{app:strategy}
    \begin{description}
    \item[Max. Rule Applications] You can set the number $N_{aut}$ of
      automatic rule applications using the slider. Even if the
      automatic strategy can still apply rules after $N_{aut}$
      applications, automatic proving stops. 
    \item[Stop At] Choose when strategy execution shall stop.
      Possible values are \texttt{Default}: strategy stops when no
      rules are applicable or the maximal number of steps is reached and
      \texttt{Unclosable}: strategy stops in all situations when
      \texttt{Default} stops but also already when the first goal is
      encountered on which no further rule is (automatically) applicable. 
      
    \item[One-Step Simplification] When this is \texttt{Enabled}, some sequences of simplification
    steps are combined into a single proof step, resulting in shorter, but less transparent proofs.

    \item[Proof splitting] Influences usage of rules branching a
      proof tree. Only rules working on formulas not on
      programs fall under the chosen policy, i.e., program rules
      causing
      splits are still applied even if splitting is switched off. The
      values are \texttt{free} (withour restrictions), \texttt{Delayed}
      (allows still splitting but prefers other rules) and \texttt{Off} (no
      splitting).

    \item[Loop treatment] This setting determines how while-loops are
treated. They can be left untouched (\texttt{None}), handled
using stated invariant contracts, or
repeatedly unrolled (\texttt{Expand}). If handled using invariants,
you can either choose the new \texttt{Loop Scope} rule (recommended),
or the legacy \texttt{Transformation}-based rule.
    
     \item[Block treatment] It is possible to specify Java blocks with contracts. 
This option allows to set how \KeY\ treats such contracts.
\begin{description}
	\item[internal contract:] If this option is set, Java blocks are replaced by their contract.
	Three properties have to be shown in the proof:
	\begin{itemize}
		\item the validity of the contract
		\item the precondition if the contract is satisfied
		\item the use case of the contract
	\end{itemize}
	
	\item[external contract:] If this option is set, Java blocks are replaced by their contract.
	Only the precondition and the use case are shown in the current proof; the validity is outsourced to its own proof obligation which can be selected in the \texttt{Proof Obligation Browser}.
	
	\item[expand:] If this option is set, Java blocks are expanded and block contracts are not used.
\end{description}

    \item[Method treatment] Methods can also be left untouched
(\texttt{None}), have their method contracts applied
(\texttt{Contracts}), or be inlined, i.e. have the method body
expanded in place (\texttt{Expand}).

  \item[Merge-point statements] Merge-point statements are annotations in the source code. When two proof branches reach the same merge-point statement, they can be merged if this setting is set to \texttt{Merge}. If it is set to \texttt{Skip}, the merge-point statements are ignored; if it is set to \texttt{None}, the automatic proof search stops when encountering a merge-point statement. For more details on merge-point statements, see \cite{Steinhoefel2019}.
      
    \item[Dependency contracts] For the simplification of heap terms setting this option to \texttt{On}
				the information in JML's \texttt{accessible} clause is used. 
% 				For instance, consider the term<br>" +
%         		             "<center><i>f(store(heap,o,a,1))</i></center>" +
%         		             "If <i>f</i> does not depend on the location <i>(o,a)</i>, which is<br>" +
%         		             "expressed by an <tt>accessible</tt> clause, then the term can be <br>" +
%         		             "simplified to <i>f(heap)
    
    
    \item[Query treatment] A query is a method that is used as a 
    function in the logic and stems from the specification.
    There are three options for query treatment in \KeY{}:
    \begin{description}
     \item[On:] Rewrite a query to a method call such that contracts 
     or inlining (dependent on the method treatment setting) can be used.
     \item[Restricted:] Same as \textsf{On} but with restrictions:
     \begin{itemize}
      \item Priority of expanding queries that occur earlier on a branch is higher than
	    for queries introduced more recently. This approximates in a breath-first search
	    with respect to query expansion.
      \item Reexpansion of identical query terms is suppressed.
      \item A query is not expanded if one of its arguments contains a literal greater
            than a computed limit\footnote{The computation of this limit is done with 
            sophisticated methods for loop detection and would go beyond the scope of this quicktour}, 
            or smaller than a computed limit. 
            This helps detecting loops in a proof
      \item Queries are expanded after the loop body in the \texttt{Preserves Invariant}
	    branch of the loop invariant rule
      \item Queries are expanded in the \texttt{Base Case} and the conclusio of the \texttt{Step Case}
	  branch when using Auto Induction
     \end{itemize}

     \item[Off:] The query statements are ignored and the proof has to be done without using them.
    \end{description}
    In addition the \kp\ offers a setting for the expansion 
    of local queries in certain safe cases. Safe cases are:
    \begin{itemize}
     \item the return type of the expanded method is known
     \item the object on which the methodcall is invoked is self or a parent of self.
    \end{itemize}
    This setting is indepedent of the query treatment setting.


%       Queries used as terms in formulas are
%       evaluated either by symbolical execution (\texttt{Expand}), or are
%       moved to the succedent (\texttt{Prog2Succ}) so that contracts can
%       be used, or are not evaluated at all (\texttt{None}).
%       Setting \texttt{Expand Local Queries} to on results in an evaluation 
%       of the queries used as terms in formulas by symbolical execution. 
%    
   \item[Arithmetic treatment] The \kp\ has several options for the treatment of arithmetic expressions:
   \begin{description}
   \item[Basic:] Using this option, polynomial expressions are simplified. 
		 In the antecedent Gr\"{o}bner Bases are computed polynomials.
		 Linear inequations are handled using (partial) Omega procedures.

   \item[DefOps:] Using the option \textsf{DefOps}, symbols such as:\\
                \texttt{/}, \texttt{\%}, \texttt{jdiv}, \texttt{jmod}, ...\\
                \texttt{int\_RANGE}, \texttt{short\_MIN}, ...\\
                \texttt{inInt}, \texttt{inByte}, ...\\
                \texttt{addJint}, \texttt{mulJshort}, ...\\
		 are expanded.
                
    \item[Model Search:] Setting the \textsf{model search} option, 
		  the \kp\ supports non-linear equations and model search.
		  Additionally multiplication of inequations with each other
		  and systematic case distinctions  (cuts) can be performed.
                This method is guaranteed to find counterexamples for
                invalid goals that only contain polynomial (in)equations.
                Such counterexamples turn up as trivially unprovable goals.
                It is also able to prove many more valid goals involving
                (in)equations, but will in general not terminate on such goals.
                
   \end{description}

   
   \item[Quantifier treatment] Sometimes quantifiers within the
      sequent have to be instantiated. This can be either done
      manually (\texttt{None}) or automatically with different
      alternatives:
      \begin{description}
        \item[\texttt{No Splits}] Instantiate a quantifier only if
          this will not cause the proof to split.
        \item[\texttt{Unrestricted}] Instantiates a quantifier even
          when causing splits. However the startegy tries to predict
          the number of caused open branches and will prefer those
          with no or only few splits.
        \item[\texttt{No Splits with Progs}] Chooses between the
          \texttt{No Splits} and \texttt{Unrestricted} behaviour
          depending on prgrams present in the sequent. If a program is
          still present, the \texttt{No splits} behavior is
          used. Otherwise, quantifiers are instantiated without restrictions.
      \end{description}
  
     \item[Class axiom rule] This setting determines how class axioms and invariants are dealt with.
     \begin{description}
     	\item[\texttt{Free}] Class axioms are expanded freely.
     	\item[\texttt{Delayed}] Class axioms are only expanded after symbolic execution, i.e., when there are no more modalities on the sequent.
     	\item[\texttt{Off}] Class axioms are never expanded automatically.
     \end{description}
 
     \item[Auto induction] This setting can allow the \kp\ to automatically perform inductive proofs for certain formulas.
     \begin{description}
     \item[\texttt{On}] Perform inductive proofs for formulas of a certain form.
     \item[\texttt{Restricted}] Perform inductive proofs for formulas of a certain form, but only if the name of the induction variable ends with \texttt{Ind} or \texttt{IND}.
     \item[\texttt{Off}] Don't perform inductive proofs automatically.
     \end{description}
       
    \end{description}   
%   \end{description}


\section{Handy Shortcuts and Buttons}
\label{app:shortcuts}
In the following an overview of all shortcuts currently used 
in the \kp\ is given. Additional, if buttons in the toolbar exist, 
their actions are listed here as well.

\begin{tabular}{llc}
\toprule
Action/Command & Shortcut & Button in Toolbar\\

\midrule
\multicolumn{3}{l}{\textbf{File related}}\\
Load & Ctrl+O & \includegraphics[width=2ex]{../figures/open}\\
Reload & Ctrl+R &\includegraphics[width=2ex]{../figures/openMostRecent}\\
Save & Ctrl+S & \includegraphics[width=2ex]{../figures/saveFile}\\
Proof Management & Ctrl+M & \includegraphics[width=10ex]{../figures/proofManagementButton}\\
Exit &Ctrl+Q& -\\
Edit last opened file & - & \includegraphics[width=2ex]{../figures/editFile}\\
\midrule
\multicolumn{3}{l}{\textbf{Appearance}}\\
Use pretty syntax & Ctrl+P & -\\
Font size: smaller & Ctrl+Up & -\\
Font size: larger& Ctrl+Down & -\\
\midrule
\multicolumn{3}{l}{\textbf{Proof specific}}\\
Start automatic strategy & Ctrl+S & \includegraphics[width=2ex]{../figures/autoModeStart}\\
Abandon task &Ctrl+W & -\\
Undo last rule application & Ctrl+Z & \includegraphics[width=2ex]{../figures/goalBack}\\
Search in proof tree & Ctrl+F & -\\
Search in sequent view & F3 & -\\
Prune tree below selected node & - & \includegraphics[width=2ex]{../figures/pruneProof} \\
SMT Solver & - &  \includegraphics[width=10ex]{../figures/SMTButton} \\
\midrule
\multicolumn{3}{l}{\textbf{Options}}\\
Taclet options &Ctrl+T&-\\
Toggle one Step Simplifier & Ctrl+Shift+S & \includegraphics[width=2ex]{../figures/oneStepSimplifier}\\
\bottomrule
\end{tabular}

\section{Setting Up Own Projects}
\label{app:configuringProjects}

If not specified otherwise via a classpath directive, \KeY\ includes a
restricted set of signatures of classes and methods from the default
standard library. The current set of classes can be found at~\url{https://git.key-project.org/key-public/key/-/tree/stable/key/key.core/src/main/resources/de/uka/ilkd/key/java/JavaRedux}.

For documentation on how to set up your own classpath, see \url{https://www.key-project.org/docs/Using\%20KeY/Classpath/}.
	

%%% Local Variables: 
%%% mode: latex
%%% TeX-master: "quicktour"
%%% End: 
