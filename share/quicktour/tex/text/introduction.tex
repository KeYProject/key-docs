\section{Introduction}
\label{sec:introduction}

When we started writing this document, we aimed at providing a short
tutorial accompanying the reader at their first steps with the \KeY\
system. The \kt\ is designed as an integrated environment for
creating, analyzing, and verifying software models and their
implementation. The reader shall learn how to install and use the
basic functionality of the \kt. Besides practical advises how to
install and get \KeY\ started, we show along a small project how to
use the \kt\ to verify programs.

Verification means proving that a program complies with its specification
in a mathematically rigorous way. In order to fulfill this task, the
specification needs to be given in a formal language with a precisely
defined meaning. In the current version of the document, we focus on
the popular \emph{Java Modeling Language}
(JML)~\cite{JMLReferenceManual11,Leavens-Baker-Ruby04} as specification
language.

In the next sections, we show how to verify a JML-annotated (specified)
Java\-Card program. For this purpose, \KeY\ features a calculus for the complete
Java\-Card language including advanced features like transactions.

Besides JML, the \kt\ supports Java\-Card\-DL as a specification
language. 

For a longer discussion on the architecture, design philosophy, and
theoretical underpinnings of the \kt\, please refer to
\cite{KeYBook2016}.
%
In case of questions or comments, don't hesitate to contact the
\KeY support team at 
\href{mailto:support@key-project.org}{\emph{support@key-project.org}}.

\subsection{Version Information}
\label{sec:version}

This tutorial was tested for \KeY\ version 2.10.

\subsection{Installation}
\label{sec:install}

You can choose between different methods to install and use \KeY.
For this tutorial, we
recommend the \javaWS\ variant described in
Sect.~\ref{install:javaws}. 

\subsubsection{The \kp{} by \javaWS}
\label{install:javaws}

\javaWS\ is a Java Technology which allows you to start applications
directly from a website. No installation is needed. You can visit our
homepage
\begin{center}
  \url{http://www.key-project.org/download}
\end{center}
which contains a link to \javaWS\ the \kp.
%
Please note that you need the \javaWS\ facility
(which should come along with your Java distribution).

\subsubsection{Byte Code and Source Code Installation}
\label{install:byteandsourcecode}

The download site offers also the binary and source-code versions of
\KeY.

%\emph{Please note:} Support for Borland Together has been discontinued
%with \KeY~1.4.
% Nevertheless the code is still present and the basic
% functionality should work with the stand-alone (non-eclipse) versions
% of Together Solo or TogetherCC, but that is without guarantee. The
% bytecode resp.\ sourcecode version is required for the Borland
% Together plug-in.
        
% NOTE: The Eclipse plugin cant be used with latest Eclipse
%
% \subsubsection{The \KeY-plugin for Eclipse}
% \label{install:eclipse}
% 
% In this section we will describe how to setup the \KeY-plugin
% for Eclipse as well as the use of some of its core features.
% We assume that Eclipse has already been installed on the
% target computer. Start it and in the menu \emph{Help} select
% \emph{Software Updates} and then \emph{Find and Install}.
% In the new window activate \emph{Search for new features to
% install} and click on \emph{Next}. Now add the \emph{New Remote
% Site} \url{http://www.key-project.org/KeYDists/KeY_Feature/}
% with a name of your choice. 
% Click on \emph{Finish}. Now a new window should appear with
% a list of installable features. Mark \KeY\ and continue.
% After accepting the license agreement and selecting a usable location
% you will be asked to verify the installation. Do so by clicking on
% \emph{Install all} and restart Eclipse when asked to. This completes
% the installation. It is possible to update the \KeY-plugin by selecting
% \emph{Search for updates of the currently installed features} in the 
% \emph{Software updates} menu.
% You can now start the standalone version of the prover by either clicking
% on the \KeY\ logo in the menu bar or from the \emph{Verification} menu.


%%% Local Variables: 
%%% mode: latex
%%% TeX-master: "quicktour"
%%% End: 
